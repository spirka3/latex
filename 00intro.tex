\chapter*{Úvod}\label{chap:intro}

% stručný úvod do problematiky, stav poznania v danej oblasti, význam a ciele práce, opäť v zrozumiteľnom jazyku bežnému odborníkovi z vášho odboru. Na konci úvodu stručne vymenovať, čo sa nachádza v jednotlivých kapitolách práce

Ako súčasť marketingovej stratégie je často použitá určitá forma reklamy na získanie povedomia o značke alebo prebiehajúcej kampani. Firmy používajú reklamy na rôznych miestach, ako napríklad v televízií, rádiu alebo sociálnych sieťach. Medzi najbežnejšie spôsoby propagácie vo vonkajších priestoroch patria reklamné plochy pri cestách. Sú umiestnené pozdĺž ciest a diaľníc, odkiaľ sú viditeľné ako pre vodičov, tak aj pre ostatných účastníkov dopravy.

Reklamný obsah je väčšinou umiestnený na bilbordoch, ktoré sú dobre viditeľné aj z väčšej vzdialenosti. Na Slovensku sú najviac zastúpené tradičné statické bilbordy. Ich obsah je na vytlačenom papieri, ktorý sa po čase vymení. Modernejšie sú elektronické bilbordy, ktoré sú z tohto pohľadu praktickejšie a predovšetkým svojou svietivosťou dokážu upútať väčšiu pozornosť vodiča v porovnaní s tradičným bilbordom. Okrem toho existujú aj iné reklamné plochy, napríklad na budovách, plotoch alebo autobusových zastávkach či dokonca na samotných vozidlách v cestnej premávke. Reklamy sú už dlhé roky bežnou súčasťou dopravy a sú vnímané ako vizuálny smog pri cestách.

To aký má reklama dosah na ľudí záleží okrem iného od typu reklamy, jej strategického umiestnenia a napokon podľa samotného obsahu. Reklamy pri cestách si bezpochyby vedia získať vodičovú pozornosť, čo je síce pozitívne pre marketing, ale na druhej strane, aj krátka strata vodičovej pozornosti môže byť príčinou dopravnej nehody s katastrofickými následkami.

Otázkou bezpečnosti reklám sa zaoberá niekoľko štúdii vrátane našej práce. Na začiatku približujeme problematiku a skúmame teoretické možnosti metód spracovania obrazu a neurónových sietí na detekciu reklamných plôch pri cestách. Prvým cieľom práce je vyvinúť čo najlepší detektor reklamných plôch, aby dokázal dostatočne presne sledovať reklamné plochy z video nahrávok. Pre všetky nájdené reklamy určujeme ich významnosť, ktorú zaraďujeme do jednej zo štyroch kategórií: slabá, nízka, stredná a vysoká. Kategória koreluje s dĺžkou vypočítanej sledovanosti danej reklamy. Druhým cieľom je vedieť klasifikovať nájdené reklamy do príslušných kategórií pomocou strojového učenia na základe viacerých príznakov, vrátane mapy význačností. V práci opisuje návrh, riešenie a napokon sa na konci venujeme celkovému vyhodnoteniu.

%\quickwordcount{main}
%\quickcharcount{main}
%There are \thechar characters and approximately \theword spaces. That makes approximately \the\numexpr\theword+\thechar\relax\ characters total.
