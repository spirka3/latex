\chapter*{Záver}\label{chap:conclusion}

% uviest co vyplyva z vysledkov z hladiska bezpecnost / marketingu + poskytnut vysledky relevantnym osobam


% Conclusions
% Visual smog may distract the driver’s attention or also influence driver’s behaviour. The most dangerous for driver are long glances at advertisements, which takes more than 0.75 seconds. In several cases the glances were longer than 2 seconds what means the driver’s distraction rapidly increases. Therefore, it is necessary to eliminate some of advertisements and after that possibility of driver’s distraction is decreasing. But at other hand, according research of driver’s EEG, the existence of billboards and road advertisements can change the progressive loss of driver attention by elimination of drowsiness. These findings are suitable for next research in condition of HMI laboratory.

% https://cyberleninka.org/article/n/1495035.pdf


% Otázkou bezpečnosti sa zaoberá niekoľko štúdii a aj samotné mestá. Vedenie sa v niektorých mestách rozhodlo znížiť či dokonca zakázať reklamné plochy s cieľom vyčistiť mesto od vizuálneho smogu a tým prispieť k bezpečnejšej doprave. Vo výnimočných prípadoch sú reklamné plochy použité na propagáciu bezpečnej jazdy, ktoré slúžia ako pripomienka bezpečnosti a podľa viacerých výskumov majú na vodičov pozitívny vplyv. 

% roadside advertising signs have evolved from static images to incorporate digital displays and changing pictures/videos designed to capture drivers’ attention. Therefore, these technological differences are likely to influence driving task demands in different ways
