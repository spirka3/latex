\chapter*{Záver}\label{chap:conclusion}

Cieľ našej práce bolo vytvorenie systému schopného detegovať vizuálny smog pri cestách. Pri hľadaní implementácie sme preštudovali rôzne možnosti spracovania obrazu, hľadania významných oblastí v obraze a prácu s neurónovými sieťami. Údaje na analýzu sme získavali počas jazdy autom v blízkosti fakulty s viacerými vodičmi čo len pridalo komplexitu k riešeniu tejto práce a zároveň nám umožnilo pracovať s dátami získanými v našom domácom prostredí.

V úvode našej práce som uviedol niektoré dôležité pojmy týkajúce sa detekcie objektov, konkrétne reklám a následnej významnosti týchto reklám. Významnosť reklamy bol dôležitý parameter, od ktorého sa odvíjal zvyšok práce. Bolo zaujímavé sledovať niektoré vlastnosti reklám a ich vplyv na celkovú významnosť.

V ďalšej časti práce sme sa venovali téme neurónových sietí a ich možnému uplatnenie v tomto odvetví. V doprave a celkovo v bežnom živote je ich potenciál zatiaľ nenaplnený, čo sa už ale pomaly mení a umožňuje im to meniť náš svet na bezpečnejší.

V návrhu riešenia a počas realizácie sme premýšľali nad možným postupom a hľadaním najvhodnejšieho možného riešenia. Postup bolo potrebné viac krát upraviť kvôli nemožnosti postupovať niektorými smermi. Dôležité bolo zabezpečenie kvalitných dát vhodných na ďalšie použitie. Na získanie príznakov sme použili klasifikátory, na základe ktorých budeme môcť vyhodnotiť vplyv vizuálneho smogu pri cestách.

V poslednej časti práce sú uvedené výsledky a závery z našej práce, ktoré ukázali, že reklamy v blízkosti ciest dokážu negatívne ovplyvniť správanie vodičov na cestách. A aj keď získajú vodičovu pozornosť len na chvíľu, tak to dokáže spôsobiť výrazne zhoršenie schopnosti vodiča reagovať na dianie na ceste.

Našimi ďalšími cieľmi by bolo ďalšie experimentovanie s detektormi a sledovanie ich výkonnosti. Na základe novozískaných dát by bolo možné vzpracovať vzor reklamy, ktorá by ovplyvňovala vodičov, čo najmenej a pritom podala potrebnú informáciu svojmu publiku.



% Hlavným cieľom našej práce je preskúmať detekciu reklamných plôch popri cestách s využitím neurónových sietí. Následne takýmto reklamám určiť ich významnosť a to potom overiť pomocou eyetrackera.

% Teoretická časť obsahuje dve časti. V prvej kapitole sa venujeme zadefinovaniu pojmov týkajúcich sa detekcie objektov a významnosti reklám. Na to nadviažeme v druhej kapitole, ktorá je teoreticky zameraná na neurónové siete a prácu s nimi. Ďalšie časti práce sa venujú praktickému riešeniu. Tretia kapitola obsahuje návrh nášho riešenia, ktoré v štvrtej kapitole realizujeme. Výstupy z našej práce sú zhrnuté v poslednej, piatej časti práce.

% hd2s sa dá trénovať

% rozsah cca 1-2 strany
% zhrnutie dosiahnutých výsledkov, v čom vidíte plusy, mínusy a váš pokrok.
% možnosti pre ďalšiu, budúcu prácu v oblasti ( otvorené problémy)

% optimalizácia hyperparametre, Under/over-fitting
% augmentacia pred trenovanim, pri testovani (Test time augmentation)
% porovnať chybosť kategorizácie, kebyže sa to počíta zo standardného modelu
% porovnať chybosť klasifikácie, kebyže sa to počíta zo standardného modelu

% Abstrakt - Úvod - Záver
% obsahujú podobné informácie
% abstrakt - kratší text, ktorý má čitateľovi pomôcť rozhodnúť sa, či si vôbec práce chce prečítať
% úvod - umožňuje zorientovať sa v práci skôr než ju začne čítať
% záver - sumarizuje najdôležitejšie veci po tom, ako prácu prečítal, môže sa teda zamerať viac na detaily a využívať pojmy zavedené v práci

% Conclusions
% Visual smog may distract the driver’s attention or also influence driver’s behaviour. The most dangerous for driver are long glances at advertisements, which takes more than 0.75 seconds. In several cases the glances were longer than 2 seconds what means the driver’s distraction rapidly increases. Therefore, it is necessary to eliminate some of advertisements and after that possibility of driver’s distraction is decreasing. But at other hand, according research of driver’s EEG, the existence of billboards and road advertisements can change the progressive loss of driver attention by elimination of drowsiness. These findings are suitable for next research in condition of HMI laboratory.

%Výsledky poskytujú informácie o možnej zníženej bezpečnosti na ceste príčinou reklamy. Okrem toho môžu slúžiť aj ako podklady pre správcov reklám pre lepšiu predstavu o dosahu reklamy k vodičom.

% https://cyberleninka.org/article/n/1495035.pdf


% Otázkou bezpečnosti sa zaoberá niekoľko štúdii a aj samotné mestá. Vedenie sa v niektorých mestách rozhodlo znížiť či dokonca zakázať reklamné plochy s cieľom vyčistiť mesto od vizuálneho smogu a tým prispieť k bezpečnejšej doprave. Vo výnimočných prípadoch sú reklamné plochy použité na propagáciu bezpečnej jazdy, ktoré slúžia ako pripomienka bezpečnosti a podľa viacerých výskumov majú na vodičov pozitívny vplyv.

% roadside advertising signs have evolved from static images to incorporate digital displays and changing pictures/videos designed to capture drivers’ attention. Therefore, these technological differences are likely to influence driving task demands in different ways